\chapter{Requisitos Não Funcionais}
\label{sec-rnfs}
\vspace{-1cm}

A Tabela~\ref{tabela-rnfs} apresenta a especificação dos requisitos não funcionais identificados no Documento de Especificação de Requisitos, os quais foram considerados condutores da arquitetura.

% Contador para IDs de Requisitos Não Funcionais.
% Substitua rnf-definir-label dentro dos \label{} abaixo por IDs do seu projeto.
\newcounter{rnfcount}
\renewcommand*\thernfcount{RNF-\arabic{rnfcount}}
\newcommand*\RNF{\refstepcounter{rnfcount}\thernfcount}
\setcounter{rnfcount}{0}

\begin{footnotesize}
	\begin{longtable}{|r|p{13cm}|}
		\caption{Especificação de Requisitos Não Funcionais.}
		\label{tabela-rnfs}
		\\\hline

		% Requisito não funcional 1.
		\multicolumn{2}{|p{\dimexpr\linewidth-2\tabcolsep-2\arrayrulewidth}|}{\cellcolor{lightgray}\RNF\label{RNF-01} -- O sistema deve controlar o acesso às funcionalidades através de login}
		\\\hline

		Categoria: & Segurança
		\\\hline

		\parbox[t]{2cm}{\raggedleft Tática /
		\\Tratamento:} & O sistema deve implementar autenticação por credenciais (login e senha) com criptografia de senhas no banco (hash, salt, variáveis de ambiente) e controle de sessão (token com dados de acesso). Cada usuário autenticado poderá acessar apenas suas próprias informações de compra. As sessões deverão expirar após período de inatividade definido (expiração de token). \\\hline

		Medida:    & Percentual de tentativas de acesso não autorizado detectadas durante testes de penetração ou auditoria.
		\\\hline

		\parbox[t]{2cm}{\raggedleft Critério de
		\\Aceitação:} & Nenhuma tentativa de acesso não autorizado deve ser bem-sucedida (0\%). O controle de acesso deve impedir completamente que um cliente tenha acesso ou altere dados de outro cliente. \\\hline

		% Linha em branco.
		\multicolumn{2}{c}{}
		\\\hline


		% Requisito não funcional 2.
		\multicolumn{2}{|p{\dimexpr\linewidth-2\tabcolsep-2\arrayrulewidth}|}{\cellcolor{lightgray}\RNF\label{RNF-02} -- A interface do usuário deve ser intuitiva e amigável. O sistema deve permitir que os usuários escolham esquemas de cores alternativos que atendam melhor às suas capacidades visuais}
		\\\hline

		Categoria: & Usabilidade
		\\\hline
		%Aplicar padrões visuais consistentes e design responsivo
		\parbox[t]{2cm}{\raggedleft Tática /
		\\Tratamento:} & A interface do sistema deve ser intuitiva, evitando muitas opções espalhadas pela tela que possam confundir o usuário ou criar poluição visual.

		As cores dos botões devem se relacionarem à ação a ser feita (verde para confirmar/salvar, vermelho para cancelar/deletar) e qualquer componente de ação (botão, seletor) deve se destacar do restante da interface.

		Qualquer fluxo mais complexo deve possuir uma legenda explicativa ou um guia de uso.
		\\\hline

		Medida:    & Monitoramento dos cliques dos usuários em cada fluxo/funcionalidade da interface.

		Monitoramento do tempo parado (interagindo com a interface, porém sem avançar para a próxima etapa) em cada fluxo/funcionalidade da interface.
		\\\hline

		\parbox[t]{2cm}{\raggedleft Critério de
		\\Aceitação:} &  O usuário não deve ter mais de 10 cliques dentro de 30 segundos em uma área ou componente que não realiza nenhuma ação.

		O usuário não deve ficar parado por mais de 2 minutos em uma etapa de um fluxo.
		\\\hline

		% Linha em branco.
		\multicolumn{2}{c}{}
		\\\hline


		% Requisito não funcional 4.
		\multicolumn{2}{|p{\dimexpr\linewidth-2\tabcolsep-2\arrayrulewidth}|}{\cellcolor{lightgray}\RNF\label{RNF-04} -- O sistema deve apresentar bom desempenho e tempo de resposta rápido}
		\\\hline

		Categoria: & Desempenho
		\\\hline

		\parbox[t]{2cm}{\raggedleft Tática /
		\\Tratamento:} & O sistema deve ser otimizado para reduzir tempos de resposta e melhorar o desempenho das operações mais utilizadas. As consultas ao banco de dados devem ser revisadas e otimizadas, evitando buscas desnecessárias ou operações custosas. Devem ser implementados mecanismos de cache para dados acessados com frequência e técnicas de paginação ou carregamento sob demanda para grandes volumes de informação. Além disso, recomenda-se o monitoramento contínuo do consumo de recursos e do tempo de execução das funcionalidades críticas, a fim de identificar e corrigir gargalos.\\\hline

		Medida:    & O desempenho será avaliado por meio de testes de carga e de estresse, monitorando o tempo médio de resposta das principais funcionalidades do sistema e a taxa de requisições processadas por segundo. Os resultados do tempo de resposta de uma requisição ou query serão obtidos através das ferramentas de monitoramento DataDog e Sentry. Também será registrada a utilização média de CPU e memória durante os testes, utilizando os gráficos fornecidos pelo provedor de serviços em nuvem (AWS).
		\\\hline

		\parbox[t]{2cm}{\raggedleft Critério de
		\\Aceitação:} &  O tempo médio de resposta das operações principais (consulta, cadastro e atualização) não deve exceder 2 segundos sob carga normal de uso e 5 segundos em situações de pico. O sistema deve suportar pelo menos 100 requisições simultâneas sem apresentar degradação significativa de desempenho ou indisponibilidade.\\\hline

		% Linha em branco.
		\multicolumn{2}{c}{}
		\\\hline

		% Requisito não funcional 5.
		\multicolumn{2}{|p{\dimexpr\linewidth-2\tabcolsep-2\arrayrulewidth}|}{\cellcolor{lightgray}\RNF\label{RNF-05} -- O sistema deve ter compatibilidade com diferentes sistemas operacionais}
		\\\hline
		%Compatibilidade/Portabilidade
		Categoria: & Compatibilidade
		\\\hline

		\parbox[t]{2cm}{\raggedleft Tática /
		\\Tratamento:} & A aplicação deve ser desenvolvida utilizando tecnologias web padrão (HTML5, CSS3, JavaScript) e frameworks modernos que garantam compatibilidade com os principais navegadores (Google Chrome, Mozilla Firefox, Microsoft Edge, Safari) em diferentes versões. Deve-se utilizar bibliotecas e ferramentas que facilitem a adaptação da interface para diferentes resoluções de tela e dispositivos.\\\hline

		Medida:    & A compatibilidade deve ser verificada por meio de testes de execução da aplicação em cada navegador-alvo, registrando-se eventuais falhas específicas de cada ambiente e versão através da ferramenta de detecção de erros Sentry.                                                                                                                                                                                                                                                                      \\\hline

		\parbox[t]{2cm}{\raggedleft Critério de
		\\Aceitação:} & As ações do usuário nunca devem resultar em erros na interface da aplicação (front-end) devido imcompatibilidades dos frameworks e bibliotecas para diferentes navegadores e suas versões. Qualquer erro detectado deve ter origem apenas nas camadas inferiores, como na lógica de negócio (que se deve ficar no backend) ou no banco de dados. \\\hline

		% Linha em branco.
		\multicolumn{2}{c}{}
		\\\hline

		% Requisito não funcional 6.
		\multicolumn{2}{|p{\dimexpr\linewidth-2\tabcolsep-2\arrayrulewidth}|}{\cellcolor{lightgray}\RNF\label{RNF-06} --O código-fonte deve ser bem documentado para facilitar a manutenção futura e devem ser implementadas boas práticas de desenvolvimento de software para facilitar a extensibilidade }
		\\\hline

		Categoria: & Manutenibilidade
		\\\hline

		\parbox[t]{2cm}{\raggedleft Tática /
		\\Tratamento:} &  O código deve ser estruturado de forma modular, com alta coesão e baixo acoplamento entre componentes, facilitando alterações e extensões futuras. Cada módulo deve conter documentação clara sobre sua função, entradas e saídas, bem como instruções de uso e integração. Devem ser adotadas convenções de codificação, padronização de nomenclaturas e padrões de projeto que favoreçam a reutilização e a testabilidade do sistema.\\\hline

		Medida:    & A manutenibilidade será avaliada por meio da revisão de código (code review), utilizando a ferramenta Sonarqube e da análise de métricas de qualidade, como o grau de acoplamento entre módulos, a complexidade e a cobertura de testes. Também será observada a porcentagem de classes e métodos documentados em relação ao total do código-fonte, bem como o tempo médio necessário para compreender e modificar uma funcionalidade existente durante atividades de manutenção.                       \\\hline

		\parbox[t]{2cm}{\raggedleft Critério de
		\\Aceitação:} & Cada módulo deve ter funções bem definidas e baixo acoplamento — alterações em um módulo não devem exigir mudanças em mais de um outro módulo.
		O código será avaliado pelo Sonarqube, uma ferramenta que detecta a complexidade de funções e métodos, como também a clareza da nomeclatura
		Uma modificação simples (como ajustar uma regra de negócio) deve poder ser compreendida e alterada em até 2 horas por outro desenvolvedor que não escreveu o código.                                                                                                                                                                                                                                                                                                                                                 \\\hline

		% Linha em branco.
		\multicolumn{2}{c}{}
		\\\hline
	\end{longtable}
\end{footnotesize}
