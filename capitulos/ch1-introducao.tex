% ==============================================================================
% Projeto de Sistema - Nome do Aluno
% Capítulo 1 - Introdução
% ==============================================================================
\chapter{Introdução}
\label{sec-intro}
\vspace{-1cm}

Este documento apresenta o projeto (\textit{design}) do sistema \emph{\imprimirtitulo}, derivado do trabalho prático da disciplina de Engenharia de Software I.

O sistema Pet Shop visa apoiar as operações e o gerenciamento de um estabelecimento de serviços e produtos para animais, abrangendo o cadastro e login de clientes e funcionários, gerenciamento de filiais, controle de estoque, agendamento de serviços, registro de vendas e emissão de relatórios.
Os objetivos do sistema incluem facilitar o atendimento ao cliente, otimizar processos internos e prover relatórios de desempenho e gestão.

Além desta introdução, este documento está organizado da seguinte forma:
a Seção~\ref{sec-plataforma} apresenta a plataforma de software utilizada na implementação do sistema;
a Seção~\ref{sec-rnfs} apresenta a especificação dos requisitos não funcionais (atributos de qualidade), definindo as táticas e o tratamento a serem dados aos atributos de qualidade considerados condutores da arquitetura;
a Seção~\ref{sec-arquitetura} apresenta a arquitetura de software; por fim,
a Seção~\ref{sec-componentes} apresenta o projeto dos componentes da arquitetura.

Este documento foi produzido por:
\begin{itemize}
	\item Álvaro Davi S. Alves \
	\item Gabriel Lima \
	\item Mateus Loss \
	\item Pedro Quedevez \

\end{itemize}

\vitor{Remover a \hl{marcação em amarelo} (comando \textbackslash hl\{\}) e comentarios do professor (comando \textbackslash vitor\{\}) de todo o documento.}
