% ==============================================================================
% Projeto de Sistema - Nome do Aluno
% Capítulo 2 - Plataforma de Desenvolvimento
% ==============================================================================
\chapter{Plataforma de Desenvolvimento}
\label{sec-plataforma}
\vspace{-1cm}


%=======================================================================================================
%			Tabela de Plataforma de Desenvolvimento e Tecnologias Utilizadas
%=======================================================================================================

A plataforma de desenvolvimento do sistema \emph{\imprimirtitulo} tem como seu principal ambiente os navegadores web modernos, como Google Chrome, Mozilla Firefox e Microsoft Edge e se baseia em tecnologias amplamente utilizadas no desenvolvimento web.

Na Tabela~\ref{tabela-plataforma} são listadas as tecnologias utilizadas no desenvolvimento da ferramenta, bem como o propósito de sua utilização.

\begin{footnotesize}
	\begin{longtable}{|p{2.5cm}|c|p{5cm}|p{5.5cm}|}
		\caption{Plataforma de Desenvolvimento e Tecnologias Utilizadas.}
		\label{tabela-plataforma}                                                                                                                                                                                        \\\hline

		\rowcolor{lightgray}
		\textbf{Tecnologia}       & \textbf{Versão} & \textbf{Descrição}                                    & \textbf{Propósito}                                                                                         \\\hline
		\endfirsthead
		\hline
		\rowcolor{lightgray}
		\textbf{Tecnologia}       & \textbf{Versão} & \textbf{Descrição}                                    & \textbf{Propósito}                                                                                         \\\hline
		\endhead

		Angular                   & 15.x.x          & Framework para TypeScript                             & Utilizado para o desenvolvimento do frontend do sistema.                                                   \\ \hline
		Java                      & 17.x.x          & Linguagem de programação                              & Utilizada para o desenvolvimento do backend do sistema.                                                    \\ \hline
		Spring Boot               & 3.x.x           & Framework para Java                                   & Utilizado para facilitar a criação de aplicações web e serviços RESTful no backend.                        \\ \hline
		PostgreSQL                & 15.x.x          & Sistema de gerenciamento de banco de dados relacional & Utilizado para armazenar os dados do sistema (statefull).                                                  \\ \hline
		Jakarta CDI               & 3.x.x           & Especificação para Injeção de Dependências em Java    & Utilizado para gerenciar a criação e o ciclo de vida dos objetos no backend.                               \\ \hline
		Jakarta Persistence (JPA) & 3.x.x           & Especificação para Persistência de Dados em Java      & Utilizado para mapear objetos Java para tabelas do banco de dados relacional (ORM).                        \\ \hline
		Hibernate                 & 6.x.x           & Framework de mapeamento objeto-relacional (ORM)       & Utilizado para implementar a persistência de dados no backend, facilitando operações com o banco de dados. \\ \hline
	\end{longtable}
\end{footnotesize}




%=======================================================================================================
%			Tabela de Softwares de Apoio ao Desenvolvimento do Projeto
%=======================================================================================================

Na Tabela~\ref{tabela-software} vemos os softwares que apoiaram o desenvolvimento de documentos e também do código fonte.

\begin{footnotesize}
	\begin{longtable}{|p{2.5cm}|c|p{5cm}|p{5.5cm}|}
		\caption{Softwares de Apoio ao Desenvolvimento do Projeto}
		\label{tabela-software}                                                                                                                                                                                        \\\hline

		\rowcolor{lightgray}
		\textbf{Tecnologia} & \textbf{Versão} & \textbf{Descrição}                                                    & \textbf{Propósito}                                                                             \\\hline
		\endfirsthead
		\hline
		\rowcolor{lightgray}
		\textbf{Tecnologia} & \textbf{Versão} & \textbf{Descrição}                                                    & \textbf{Propósito}                                                                             \\\hline
		\endhead

		Visual Studio Code  & 1.105.1         & Editor de código-fonte com suporte a várias linguagens de programação & Utilizado para o desenvolvimento do código fonte do sistema.                                   \\ \hline
		\LaTeX              & TeX Live 2023   & Sistema de compilação de documentos                                   & Utilizado para a produção do documento de projeto do sistema.                                  \\ \hline
		Overleaf            & -               & Plataforma online para edição de documentos em \LaTeX                 & Utilizado para a produção colaborativa do documento de projeto do sistema.                     \\ \hline
		Draw.io             & -               & Ferramenta online para criação de diagramas UML                       & Utilizado para a criação dos diagramas UML presentes no documento.                             \\ \hline
		Git                 & -               & Sistema de controle de versão distribuído                             & Utilizado para o versionamento do código fonte do sistema.                                     \\ \hline
		GitHub              & -               & Plataforma de hospedagem de código-fonte                              & Utilizado para controle de versão e colaboração no desenvolvimento do código fonte do sistema. \\ \hline
	\end{longtable}
\end{footnotesize}
