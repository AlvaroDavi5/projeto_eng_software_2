% ==============================================================================
% Projeto de Sistema - Nome do Aluno
% Capítulo 2 - Plataforma de Desenvolvimento
% ==============================================================================
\chapter{Plataforma de Desenvolvimento}
\label{sec-plataforma}
\vspace{-1cm}


%=======================================================================================================
%			Tabela de Plataforma de Desenvolvimento e Tecnologias Utilizadas
%=======================================================================================================

A plataforma de desenvolvimento do sistema \emph{\imprimirtitulo} tem como seu principal ambiente os navegadores web modernos, como Google Chrome, Mozilla Firefox e Microsoft Edge e se baseia em tecnologias amplamente utilizadas no desenvolvimento web.

Na Tabela~\ref{tabela-plataforma} são listadas as tecnologias utilizadas no desenvolvimento da ferramenta, bem como o propósito de sua utilização.

\begin{footnotesize}
	\begin{longtable}{|p{2.5cm}|c|p{5cm}|p{5.5cm}|}
		\caption{Plataforma de Desenvolvimento e Tecnologias Utilizadas.}
		\label{tabela-plataforma}                                                                                                                                                                                                                                                            \\\hline

		\rowcolor{lightgray}
		\textbf{Tecnologia}       & \textbf{Versão} & \textbf{Descrição}                                                              & \textbf{Propósito}                                                                                                                                   \\\hline
		\endfirsthead
		\hline
		\rowcolor{lightgray}
		\textbf{Tecnologia}       & \textbf{Versão} & \textbf{Descrição}                                                              & \textbf{Propósito}                                                                                                                                   \\\hline
		\endhead

		TypeScript                & 5.x.x           & Linguagem de programação baseada em JavaScript com suporte a tipagem estática   & Utilizada para o desenvolvimento do frontend do sistema.                                                                                             \\ \hline
		HTML                      & 5               & Linguagem de marcação para estruturação de páginas web                          & Utilizada para estruturar o conteúdo das páginas web do sistema.                                                                                     \\ \hline
		CSS                       & 3               & Linguagem de estilo para páginas web                                            & Utilizada para estilizar e formatar a aparência das páginas web do sistema.                                                                          \\ \hline
		Angular                   & 15.x.x          & Framework para TypeScript que utiliza HTML+CSS+TS                               & Utilizado para o desenvolvimento do frontend do sistema para renderização de páginas web estáticas e dinâmicas.                                      \\ \hline
		Microsoft Clarity         & -               & Ferramenta de análise de comportamento do usuário em websites                   & Utilizada para monitorar e analisar o comportamento dos usuários na interface do sistema.                                                            \\ \hline
		Java                      & 17.x.x          & Linguagem de programação                                                        & Utilizada para o desenvolvimento do backend do sistema.                                                                                              \\ \hline
		Maven                     & 3.x.x           & Ferramenta de automação de compilação e gerenciamento de dependências para Java & Utilizada para gerenciar as dependências e o ciclo de vida do projeto Java no backend.                                                               \\ \hline
		Spring Boot               & 3.x.x           & Framework para Java                                                             & Utilizado para facilitar a criação de aplicações web e serviços RESTful no backend.                                                                  \\ \hline
		PostgreSQL                & 15.x.x          & Sistema de gerenciamento de banco de dados relacional                           & Utilizado para armazenar os dados do sistema (statefull).                                                                                            \\ \hline
		Jakarta Persistence (JPA) & 3.x.x           & Especificação para Persistência de Dados em Java                                & Utilizado para mapear objetos Java para tabelas do banco de dados relacional (ORM).                                                                  \\ \hline
		Hibernate                 & 6.x.x           & Framework de mapeamento objeto-relacional (ORM)                                 & Utilizado para implementar a persistência de dados no backend, facilitando operações com o banco de dados.                                           \\ \hline
		AWS                       & -               & Plataforma de serviços em nuvem                                                 & Utilizada para hospedar a aplicação, banco de dados e outros serviços necessários para o funcionamento do sistema.                                   \\ \hline
		Sentry                    & -               & Plataforma de monitoramento de erros e desempenho                               & Utilizada para rastrear e monitorar erros e exceções na aplicação em produção, além do desempenho de uma ação ou funcionalidade.                     \\ \hline
		DataDog                   & -               & Plataforma de monitoramento e análise de desempenho                             & Utilizada para monitorar o desempenho e a saúde da aplicação em produção.                                                                            \\ \hline
		Sonarqube                 & -               & Plataforma de análise contínua de código-fonte                                  & Utilizada para garantir a qualidade do código-fonte, identificando problemas como brechas de segurança, complexidade, ineligibilidade, entre outros. \\ \hline
	\end{longtable}
\end{footnotesize}




%=======================================================================================================
%			Tabela de Softwares de Apoio ao Desenvolvimento do Projeto
%=======================================================================================================

Na Tabela~\ref{tabela-software} vemos os softwares que apoiaram o desenvolvimento de documentos e também do código fonte.

\begin{footnotesize}
	\begin{longtable}{|p{2.5cm}|c|p{5cm}|p{5.5cm}|}
		\caption{Softwares de Apoio ao Desenvolvimento do Projeto}
		\label{tabela-software}                                                                                                                                                                                                      \\\hline

		\rowcolor{lightgray}
		\textbf{Tecnologia}               & \textbf{Versão} & \textbf{Descrição}                                                    & \textbf{Propósito}                                                                             \\\hline
		\endfirsthead
		\hline
		\rowcolor{lightgray}
		\textbf{Tecnologia}               & \textbf{Versão} & \textbf{Descrição}                                                    & \textbf{Propósito}                                                                             \\\hline
		\endhead

		Visual Studio Code                & 1.105.1         & Editor de código-fonte com suporte a várias linguagens de programação & Utilizado para o desenvolvimento do código fonte do sistema.                                   \\ \hline
		\LaTeX                            & TeX Live 2023   & Sistema de compilação de documentos                                   & Utilizado para a produção do documento de projeto do sistema.                                  \\ \hline
		Overleaf                          & -               & Plataforma online para edição de documentos em \LaTeX                 & Utilizado para a produção colaborativa do documento de projeto do sistema.                     \\ \hline
		Draw.io                           & -               & Ferramenta online para criação de diagramas UML                       & Utilizado para a criação dos diagramas UML presentes no documento.                             \\ \hline
		Visual Paradigm Community Edition & 17.3.x          & Ferramenta para modelagem UML                                         & Utilizado para a criação dos diagramas UML presentes no documento.                             \\ \hline
		Git                               & -               & Sistema de controle de versão distribuído                             & Utilizado para o versionamento do código fonte do sistema.                                     \\ \hline
		GitHub                            & -               & Plataforma de hospedagem de código-fonte                              & Utilizado para controle de versão e colaboração no desenvolvimento do código fonte do sistema. \\ \hline
	\end{longtable}
\end{footnotesize}

\newpage
