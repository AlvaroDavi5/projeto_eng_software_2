
\chapter{Projeto dos Componentes da Arquitetura}
\label{sec-componentes}
\vspace{-1cm}

Conforme mostrado na Figura~\ref{figura-arquitetura} ...

\vitor{A partir da arquitetura da Figura~\ref{figura-arquitetura}, descrever no parágrafo acima os subsistemas e as camadas de cada subsistema que serão apresentadas, ajustando as subseções abaixo conforme necessário e preenchendo-as com modelos e explicações textuais. Como exemplo, vide documentos de projeto do Marvin (caso em que há apenas 1 subsistema) e da Locadora de Carros do prof. Ricardo Falbo (caso em que há mais de um subsistema, como prevê este template).}


\section{Subsistema SS01}
\label{sec-componentes-ss01}

O subsistema SS01 ...

\vitor{Descrever o escopo do subsistema.}


\subsection{Camada de Negócio}
\label{sec-componentes-ss01-negocio}

\vitor{Apresentar os modelos e descrever no texto.}


\subsection{Camada de Apresentação}
\label{sec-componentes-ss01-apresentacao}

\vitor{Apresentar os modelos e descrever no texto.}


\subsection{Camada de Acesso a Dados}
\label{sec-componentes-ss01-dados}

\vitor{Apresentar os modelos e descrever no texto.}




\section{Subsistema SS02}
\label{sec-componentes-ss02}

O subsistema SS02 ...

\vitor{Descrever o escopo do subsistema.}


\subsection{Camada de Negócio}
\label{sec-componentes-ss02-negocio}

\vitor{Apresentar os modelos e descrever no texto.}


\subsection{Camada de Apresentação}
\label{sec-componentes-ss02-apresentacao}

\vitor{Apresentar os modelos e descrever no texto.}


\subsection{Camada de Acesso a Dados}
\label{sec-componentes-ss02-dados}

\vitor{Apresentar os modelos e descrever no texto.}