
\chapter{Projeto dos Componentes da Arquitetura}
\label{sec-componentes}
\vspace{-1cm}

Conforme mostrado na Figura~\ref{figura-arquitetura}, o sistema é separado em camadas e seus pacotes, cada pacote contém seus proprios componentes, que apenas se relacionam com componentes do mesmo pacote ou com componentes de outros pacotes através de interfaces bem definidas.
A seguir, são apresentados os componentes de cada camada do sistema.

\section{Camada de Lógica de Negócio (CLN)}
\label{sec-componentes-cln}

Os componentes da Camada de Lógica de Negócio (CLN) são responsáveis por implementar as regras de negócio do sistema, realizando o processamento dos dados recebidos da camada de apresentação (CIU) e interagindo com a Camada de gerência de dados (CGD) para persistência e recuperação das informações necessárias.

A CLN também contém dois pacotes principais, o pacote Domain e o pacote Application, e interage com a Camada de Domínio do Problema (CDP) através das classes do pacote Domain, para utilizar as entidades específicas do domínio e seus métodos e atributos.

\subsection{Pacote Application}
\label{sec-componentes-cln-application}

O pacote Application é responsável por orquestrar as operações do sistema, coordenando a interação entre os componentes da CLN e a CGD. Ele contém os serviços que implementam as funcionalidades principais do sistema, utilizando as entidades do pacote Domain para realizar as operações de negócio.


\subsection{Pacote Domain}
\label{sec-componentes-cln-domain}

\begin{figure}[h]
	\centering
	\includegraphics[width=1\textwidth]{figuras/diagrama-dominio.jpg}
	\caption{Componentes de Domínio do Problema}
	\label{figura-cln-domain}
\end{figure}

O pacote Domain contém as entidades do domínio do problema, representando os conceitos e objetos relevantes para o sistema. Cada entidade encapsula seus atributos e comportamentos (métodos), fornecendo uma interface clara para a manipulação dos dados relacionados ao domínio.

\vitor{Registrar as principais decisões tomadas para se chegar a este modelo relativas a alterações estruturais do modelo, e.g., aquelas que levaram à criação de novas classes e associações.}
