% ==============================================================================
% Modelo para Especificação de Projeto de Software
% Prof. Vítor E. Silva Souza - NEMO/UFES :: DI/UFES :: PPGI/UFES
%
% Baseado em abtex2-modelo-trabalho-academico.tex, v-1.9.2 laurocesar
% Copyright 2012-2014 by abnTeX2 group at http://abntex2.googlecode.com/ 
%
% This work may be distributed and/or modified under the conditions of the LaTeX 
% Project Public License, either version 1.3 of this license or (at your option) 
% any later version. The latest version of this license is in
% http://www.latex-project.org/lppl.txt.
%
% IMPORTANTE:
% Instruções encontram-se espalhadas pelo documento. Para facilitar sua leitura,
% tais instruções são precedidas por (*) -- utilize a função localizar do seu
% editor para passar por todas elas.
% ==============================================================================

% Usa o estilo abntex2, configurando detalhes de formatação e hifenização.
\documentclass[
	12pt,				
	oneside,		
	a4paper,			
	english,			% Idioma adicional para hifenização.
	french,				% Idioma adicional para hifenização.
	spanish,			% Idioma adicional para hifenização.
	brazil				% O último idioma é o principal do documento.
	]{abntex2}


%%% Importação de pacotes. %%%

% Conserta o erro "No room for a new \count". 
% O comando \reserveinserts deve ser comentado ou não, dependendo da versão do LaTeX.
\usepackage{etex}
%\reserveinserts{28}

% Usa a fonte Latin Modern.
\usepackage{lmodern}

% Seleção de códigos de fonte.
\usepackage[T1]{fontenc}

% Codificação do documento em Unicode.
\usepackage[utf8]{inputenc}

% Usado pela ficha catalográfica.
\usepackage{lastpage}

% Indenta o primeiro parágrafo de cada seção.
\usepackage{indentfirst}

% Controle das cores.
\usepackage[usenames,dvipsnames]{xcolor}
\usepackage{colortbl}

% Inclusão de gráficos.
\usepackage{graphicx}

% Tabularx package: melhor controle de leiaute de tabelas.
\usepackage{tabularx}

% Longtable package: tabelas que passam do limite de uma página.
\usepackage{longtable}
\usepackage{pdflscape}

% Inclusão de páginas em PDF diretamente no documento (para uso nos apêndices).
\usepackage{pdfpages}

% Para melhorias de justificação.
\usepackage{microtype}

% Citações padrão ABNT.
\usepackage[brazilian,hyperpageref]{backref}
\usepackage[alf]{abntex2cite}	
\renewcommand{\backrefpagesname}{Citado na(s) página(s):~}		% Usado sem a opção hyperpageref de backref.
\renewcommand{\backref}{}										% Texto padrão antes do número das páginas.
\renewcommand*{\backrefalt}[4]{									% Define os textos da citação.
	\ifcase #1
		Nenhuma citação no texto.
	\or
		Citado na página #2.
	\else
		Citado #1 vezes nas páginas #2.
	\fi}

% \rm is deprecated and should not be used in a LaTeX2e document
% http://tex.stackexchange.com/questions/151897/always-textrm-never-rm-a-counterexample
\renewcommand{\rm}{\textrm}

% Inclusão de símbolos não padrão.
\usepackage{amssymb}
\usepackage{eurosym}

% Para utilizar \eqref para referenciar equações.
\usepackage{amsmath}

% Permite mostrar figuras muito largas em modo paisagem com \begin{sidewaysfigure} ao invés de \begin{figure}.
\usepackage{rotating}

% Permite customizar listas enumeradas/com marcadores.
\usepackage{enumitem}

% Permite inserir hiperlinks com \url{}.
\usepackage{bigfoot}
\usepackage{hyperref}

% Permite usar o comando \hl{} para evidenciar texto com fundo amarelo. Útil para chamar atenção a itens a fazer.
\usepackage{soulutf8}

% Colorinlistoftodos package: to insert colored comments so authors can collaborate on the content.
% (*) Indicar o nome do aluno e substituir o nome do professor se for o caso.
\usepackage[colorinlistoftodos, textwidth=20mm, textsize=footnotesize]{todonotes}
\newcommand{\vitor}[1]{\todo[author=\textbf{Vitor Souza},color=yellow!30,caption={},inline]{#1}}
\newcommand{\alvaro}[1]{\todo[author=\textbf{Álvaro Alves},color=green!30,caption={},inline]{#1}}
\newcommand{\gabriel}[1]{\todo[author=\textbf{Gabriel Lima},color=red!30,caption={},inline]{#1}}
\newcommand{\mateus}[1]{\todo[author=\textbf{Mateus Loss},color=blue!30,caption={},inline]{#1}}
\newcommand{\pedro}[1]{\todo[author=\textbf{Pedro Quedevez},color=orange!30,caption={},inline]{#1}}

% Permite inserir espaço em branco condicional (incluído no texto final só se necessário) em macros.
\usepackage{xspace}

% Permite incluir listagens de código com o comando \lstinputlisting{}.
\usepackage{listings}
\usepackage{caption}
\DeclareCaptionFont{white}{\color{white}}
\DeclareCaptionFormat{listing}{\colorbox{gray}{\parbox{\textwidth}{#1#2#3}}}
\captionsetup[lstlisting]{format=listing,labelfont=white,textfont=white}
\renewcommand{\lstlistingname}{Listagem}
\definecolor{mygray}{rgb}{0.5,0.5,0.5}
\lstset{
	basicstyle=\scriptsize,
	breaklines=true,
	numbers=left,
	numbersep=5pt,
	numberstyle=\tiny\color{mygray}, 
	rulecolor=\color{black},
	showstringspaces=false,
	tabsize=2,
    inputencoding=utf8,
    extendedchars=true,
    literate=%
    {é}{{\'{e}}}1
    {è}{{\`{e}}}1
    {ê}{{\^{e}}}1
    {ë}{{\¨{e}}}1
    {É}{{\'{E}}}1
    {Ê}{{\^{E}}}1
    {û}{{\^{u}}}1
    {ù}{{\`{u}}}1
    {â}{{\^{a}}}1
    {à}{{\`{a}}}1
    {á}{{\'{a}}}1
    {ã}{{\~{a}}}1
    {Á}{{\'{A}}}1
    {Â}{{\^{A}}}1
    {Ã}{{\~{A}}}1
    {ç}{{\c{c}}}1
    {Ç}{{\c{C}}}1
    {õ}{{\~{o}}}1
    {ó}{{\'{o}}}1
    {ô}{{\^{o}}}1
    {Õ}{{\~{O}}}1
    {Ó}{{\'{O}}}1
    {Ô}{{\^{O}}}1
    {î}{{\^{i}}}1
    {Î}{{\^{I}}}1
    {í}{{\'{i}}}1
    {Í}{{\~{Í}}}1
}

% Atualização ABNT para referências bibliográficas.
\bibliographystyle{abntex2-alf.bst}




%%% Definição de variáveis. %%%
% (*) Substituir os textos abaixo com as informações apropriadas.
\titulo{Pet Shop}
\local{Vitória, ES}
\data{\the\year}
\instituicao{
	Universidade Federal do Espírito Santo -- UFES
	\par
	Centro Tecnológico
	\par
	Departamento de Informática}
\newcommand{\subtitulo}{Documento de Projeto de Sistema}
\newcommand{\versao}{1.0}

% Define a capa.
\renewcommand{\imprimircapa}{%
	\begin{capa}%
		\center
		
		{\ABNTEXchapterfont\large\subtitulo{}}
		\vfill
		\begin{center}
			\ABNTEXchapterfont\bfseries\LARGE\imprimirtitulo
		\end{center}
		
		\vfill
		\large\imprimirlocal
		\linebreak
		\large\imprimirdata
		\vspace*{1cm}
	\end{capa}
}

% Macros específicas do trabalho.
% (*) Inclua aqui termos que são utilizados muitas vezes e que demandam formatação especial.
% Exemplo: Java com TM (trademark) em superscript.
% Use sempre \xspace para que o LaTeX inclua espaço em branco após a macro somente quando necessário.
\newcommand{\java}{Java\texttrademark\xspace}




%%% Configurações finais de aparência. %%%

% Altera o aspecto de algumas cores.
\definecolor{blue}{RGB}{41,5,195}
\definecolor{lightgray}{gray}{0.9}

% Informações do PDF.
\makeatletter
\hypersetup{
	pdftitle={\@title}, 
	pdfauthor={\@author},
	pdfsubject={\imprimirpreambulo},
	pdfcreator={LaTeX with abnTeX2},
	pdfkeywords={abnt}{latex}{abntex}{abntex2}{trabalho acadêmico}, 
	colorlinks=true,				% Colore os links (ao invés de usar caixas).
	linkcolor=blue,					% Cor dos links.
	citecolor=blue,					% Cor dos links na bibliografia.
	filecolor=magenta,				% Cor dos links de arquivo.
	urlcolor=blue,					% Cor das URLs.
	bookmarksdepth=4
}
\makeatother

% Espaçamentos entre linhas e parágrafos.
\setlength{\parindent}{1.3cm}
\setlength{\parskip}{0.2cm}



%%% Páginas iniciais do documento: capa, folha de rosto, ficha, resumo, tabelas, etc. %%%

% Compila o índice.
\makeindex

% Inicia o documento.
\begin{document}

% Retira espaço extra obsoleto entre as frases.
\frenchspacing

% Inclui o brasão da UFES.
\begin{figure}[h]
	\centering
	\includegraphics[scale=0.055]{brasao.jpg}
	\label{ppts3}
\end{figure}

% Capa do trabalho.
\imprimircapa



% (*) Incluir linhas no registro de alterações a cada nova versão.
\begin{center}
	{\large\bfseries Registro de Alterações:}

	\vspace{0.5cm}
	\begin{tabular}{|c|p{45mm}|c|p{60mm}|} \hline

		\textbf{Versão} & \textbf{Responsável} & \textbf{Data} & \textbf{Alterações}                                                                                                          \\ \hline

		1.0             & Mateus Loss          & 19/10/2025    & Versão inicial da arquitetura e introdução (Cap. \ref{sec-intro}).                                                           \\\hline
		1.1             & Pedro Quedevez       & 23/10/2025    & Inicio da especificação dos requisitos não funcionais (Cap. \ref{sec-rnfs}).                                                 \\\hline
		1.2             & Gabriel Lima         & 23/10/2025    & Especificação dos requisitos não funcionais (Cap. \ref{sec-rnfs}) e das camadas da arquitetura (Cap. \ref{sec-arquitetura}). \\\hline
		1.3             & Álvaro Alves         & 24/10/2025    & Especificação da plataforma e tecnologias (Cap. \ref{sec-plataforma}).                                                       \\\hline
		1.4             & Álvaro Alves         & 26/10/2025    & Construção inicial da arquitetura (Cap. \ref{sec-arquitetura}).                                                              \\\hline
		2.0             & Álvaro Alves         & 27/10/2025    & Finalização da arquitetura (Cap. \ref{sec-arquitetura}).                                                                     \\\hline
		2.1             & Álvaro Alves         & 06/11/2025    & Correções dos Cap. \ref{sec-intro} a Cap. \ref{sec-rnfs} após avaliação.                                                     \\\hline
		3.0             & Pedro Quedevez       & 09/11/2025    & Projeto dos Componentes da Arquitetura (Cap. \ref{sec-componentes}).                                                         \\\hline
		4.0             & Álvaro Alves         & 12/12/2025    & Alteração do Diagrama de Pacotes e descrição das camadas da arquitetura (Cap. \ref{sec-arquitetura}).                        \\\hline
	\end{tabular}
\end{center}
\newpage



%%% Início da parte de conteúdo do documento. %%%
% Marca o início dos elementos textuais.
\textual

% Inclusão dos capítulos.
\begingroup
\let\clearpage\relax
% ==============================================================================
% Projeto de Sistema - Nome do Aluno
% Capítulo 1 - Introdução
% ==============================================================================
\chapter{Introdução}
\label{sec-intro}
\vspace{-1cm}

Este documento apresenta o projeto (\textit{design}) do sistema \emph{\imprimirtitulo}, derivado do trabalho prático da disciplina de Engenharia de Software I.

O sistema Pet Shop visa apoiar as operações e o gerenciamento de um estabelecimento de serviços e produtos para animais, abrangendo o cadastro e login de clientes e funcionários, gerenciamento de filiais, controle de estoque, agendamento de serviços, registro de vendas e emissão de relatórios.
Os objetivos do sistema incluem facilitar o atendimento ao cliente, otimizar processos internos e prover relatórios de desempenho e gestão.

Além desta introdução, este documento está organizado da seguinte forma:
a Seção~\ref{sec-plataforma} apresenta a plataforma de software utilizada na implementação do sistema;
a Seção~\ref{sec-rnfs} apresenta a especificação dos requisitos não funcionais (atributos de qualidade), definindo as táticas e o tratamento a serem dados aos atributos de qualidade considerados condutores da arquitetura;
a Seção~\ref{sec-arquitetura} apresenta a arquitetura de software; por fim,
a Seção~\ref{sec-componentes} apresenta o projeto dos componentes da arquitetura.

Este documento foi produzido por:
\begin{itemize}
	\item Álvaro Davi S. Alves \
	\item Gabriel Lima \
	\item Mateus Loss \
	\item Pedro Quedevez \

\end{itemize}

\vitor{Remover a \hl{marcação em amarelo} (comando \textbackslash hl\{\}) e comentarios do professor (comando \textbackslash vitor\{\}) de todo o documento.}

\vspace*{1.3cm}
% ==============================================================================
% Projeto de Sistema - Nome do Aluno
% Capítulo 2 - Plataforma de Desenvolvimento
% ==============================================================================
\chapter{Plataforma de Desenvolvimento}
\label{sec-plataforma}
\vspace{-1cm}


%=======================================================================================================
%			Tabela de Plataforma de Desenvolvimento e Tecnologias Utilizadas
%=======================================================================================================

A plataforma de desenvolvimento do sistema \emph{\imprimirtitulo} tem como seu principal ambiente os navegadores web modernos, como Google Chrome, Mozilla Firefox e Microsoft Edge e se baseia em tecnologias amplamente utilizadas no desenvolvimento web.

Na Tabela~\ref{tabela-plataforma} são listadas as tecnologias utilizadas no desenvolvimento da ferramenta, bem como o propósito de sua utilização.

\begin{footnotesize}
	\begin{longtable}{|p{2.5cm}|c|p{5cm}|p{5.5cm}|}
		\caption{Plataforma de Desenvolvimento e Tecnologias Utilizadas.}
		\label{tabela-plataforma}                                                                                                                                                                                        \\\hline

		\rowcolor{lightgray}
		\textbf{Tecnologia}       & \textbf{Versão} & \textbf{Descrição}                                    & \textbf{Propósito}                                                                                         \\\hline
		\endfirsthead
		\hline
		\rowcolor{lightgray}
		\textbf{Tecnologia}       & \textbf{Versão} & \textbf{Descrição}                                    & \textbf{Propósito}                                                                                         \\\hline
		\endhead

		Angular                   & 15.x.x          & Framework para TypeScript                             & Utilizado para o desenvolvimento do frontend do sistema.                                                   \\ \hline
		Java                      & 17.x.x          & Linguagem de programação                              & Utilizada para o desenvolvimento do backend do sistema.                                                    \\ \hline
		Spring Boot               & 3.x.x           & Framework para Java                                   & Utilizado para facilitar a criação de aplicações web e serviços RESTful no backend.                        \\ \hline
		PostgreSQL                & 15.x.x          & Sistema de gerenciamento de banco de dados relacional & Utilizado para armazenar os dados do sistema (statefull).                                                  \\ \hline
		Jakarta CDI               & 3.x.x           & Especificação para Injeção de Dependências em Java    & Utilizado para gerenciar a criação e o ciclo de vida dos objetos no backend.                               \\ \hline
		Jakarta Persistence (JPA) & 3.x.x           & Especificação para Persistência de Dados em Java      & Utilizado para mapear objetos Java para tabelas do banco de dados relacional (ORM).                        \\ \hline
		Hibernate                 & 6.x.x           & Framework de mapeamento objeto-relacional (ORM)       & Utilizado para implementar a persistência de dados no backend, facilitando operações com o banco de dados. \\ \hline
	\end{longtable}
\end{footnotesize}




%=======================================================================================================
%			Tabela de Softwares de Apoio ao Desenvolvimento do Projeto
%=======================================================================================================

Na Tabela~\ref{tabela-software} vemos os softwares que apoiaram o desenvolvimento de documentos e também do código fonte.

\begin{footnotesize}
	\begin{longtable}{|p{2.5cm}|c|p{5cm}|p{5.5cm}|}
		\caption{Softwares de Apoio ao Desenvolvimento do Projeto}
		\label{tabela-software}                                                                                                                                                                                        \\\hline

		\rowcolor{lightgray}
		\textbf{Tecnologia} & \textbf{Versão} & \textbf{Descrição}                                                    & \textbf{Propósito}                                                                             \\\hline
		\endfirsthead
		\hline
		\rowcolor{lightgray}
		\textbf{Tecnologia} & \textbf{Versão} & \textbf{Descrição}                                                    & \textbf{Propósito}                                                                             \\\hline
		\endhead

		Visual Studio Code  & 1.105.1         & Editor de código-fonte com suporte a várias linguagens de programação & Utilizado para o desenvolvimento do código fonte do sistema.                                   \\ \hline
		\LaTeX              & TeX Live 2023   & Sistema de compilação de documentos                                   & Utilizado para a produção do documento de projeto do sistema.                                  \\ \hline
		Overleaf            & -               & Plataforma online para edição de documentos em \LaTeX                 & Utilizado para a produção colaborativa do documento de projeto do sistema.                     \\ \hline
		Draw.io             & -               & Ferramenta online para criação de diagramas UML                       & Utilizado para a criação dos diagramas UML presentes no documento.                             \\ \hline
		Git                 & -               & Sistema de controle de versão distribuído                             & Utilizado para o versionamento do código fonte do sistema.                                     \\ \hline
		GitHub              & -               & Plataforma de hospedagem de código-fonte                              & Utilizado para controle de versão e colaboração no desenvolvimento do código fonte do sistema. \\ \hline
	\end{longtable}
\end{footnotesize}

\vspace*{1.3cm}
\chapter{Requisitos Não Funcionais}
\label{sec-rnfs}
\vspace{-1cm}

A Tabela~\ref{tabela-rnfs} apresenta a especificação dos requisitos não funcionais identificados no Documento de Especificação de Requisitos, os quais foram considerados condutores da arquitetura.

% Contador para IDs de Requisitos Não Funcionais.
% Substitua rnf-definir-label dentro dos \label{} abaixo por IDs do seu projeto.
\newcounter{rnfcount}
\renewcommand*\thernfcount{RNF-\arabic{rnfcount}}
\newcommand*\RNF{\refstepcounter{rnfcount}\thernfcount}
\setcounter{rnfcount}{0}

\begin{footnotesize}
	\begin{longtable}{|r|p{13cm}|}
		\caption{Especificação de Requisitos Não Funcionais.}
		\label{tabela-rnfs}
		\\\hline

		% Requisito não funcional 1.
		\multicolumn{2}{|p{\dimexpr\linewidth-2\tabcolsep-2\arrayrulewidth}|}{\cellcolor{lightgray}\RNF\label{RNF-01} -- O sistema deve controlar o acesso às funcionalidades através de login}
		\\\hline

		Categoria: & Segurança
		\\\hline

		\parbox[t]{2cm}{\raggedleft Tática /
		\\Tratamento:} & O sistema deve implementar autenticação por credenciais (login e senha) com criptografia de senhas no banco (hash, salt, variáveis de ambiente) e controle de sessão (token com dados de acesso). Cada usuário autenticado poderá acessar apenas suas próprias informações de compra. As sessões deverão expirar após período de inatividade definido (expiração de token). \\\hline

		Medida:    & Percentual de tentativas de acesso não autorizado detectadas durante testes de penetração ou auditoria.
		\\\hline

		\parbox[t]{2cm}{\raggedleft Critério de
		\\Aceitação:} & Nenhuma tentativa de acesso não autorizado deve ser bem-sucedida (0\%). O controle de acesso deve impedir completamente que um cliente tenha acesso ou altere dados de outro cliente. \\\hline

		% Linha em branco.
		\multicolumn{2}{c}{}
		\\\hline


		% Requisito não funcional 2.
		\multicolumn{2}{|p{\dimexpr\linewidth-2\tabcolsep-2\arrayrulewidth}|}{\cellcolor{lightgray}\RNF\label{RNF-02} -- A interface do usuário deve ser intuitiva e amigável. O sistema deve permitir que os usuários escolham esquemas de cores alternativos que atendam melhor às suas capacidades visuais}
		\\\hline

		Categoria: & Usabilidade
		\\\hline
		%Aplicar padrões visuais consistentes e design responsivo
		\parbox[t]{2cm}{\raggedleft Tática /
		\\Tratamento:} & A interface do sistema deve ser intuitiva, evitando muitas opções espalhadas pela tela que possam confundir o usuário ou criar poluição visual.

		As cores dos botões devem se relacionarem à ação a ser feita (verde para confirmar/salvar, vermelho para cancelar/deletar) e qualquer componente de ação (botão, seletor) deve se destacar do restante da interface.

		Qualquer fluxo mais complexo deve possuir uma legenda explicativa ou um guia de uso.
		\\\hline

		Medida:    & Monitoramento dos cliques dos usuários em cada fluxo/funcionalidade da interface.

		Monitoramento do tempo parado (interagindo com a interface, porém sem avançar para a próxima etapa) em cada fluxo/funcionalidade da interface.
		\\\hline

		\parbox[t]{2cm}{\raggedleft Critério de
		\\Aceitação:} &  O usuário não deve ter mais de 10 cliques dentro de 30 segundos em uma área ou componente que não realiza nenhuma ação.

		O usuário não deve ficar parado por mais de 2 minutos em uma etapa de um fluxo.
		\\\hline

		% Linha em branco.
		\multicolumn{2}{c}{}
		\\\hline


		% Requisito não funcional 4.
		\multicolumn{2}{|p{\dimexpr\linewidth-2\tabcolsep-2\arrayrulewidth}|}{\cellcolor{lightgray}\RNF\label{RNF-04} -- O sistema deve apresentar bom desempenho e tempo de resposta rápido}
		\\\hline

		Categoria: & Desempenho
		\\\hline

		\parbox[t]{2cm}{\raggedleft Tática /
		\\Tratamento:} & O sistema deve ser otimizado para reduzir tempos de resposta e melhorar o desempenho das operações mais utilizadas. As consultas ao banco de dados devem ser revisadas e otimizadas, evitando buscas desnecessárias ou operações custosas. Devem ser implementados mecanismos de cache para dados acessados com frequência e técnicas de paginação ou carregamento sob demanda para grandes volumes de informação. Além disso, recomenda-se o monitoramento contínuo do consumo de recursos e do tempo de execução das funcionalidades críticas, a fim de identificar e corrigir gargalos.\\\hline

		Medida:    & O desempenho será avaliado por meio de testes de carga e de estresse, monitorando o tempo médio de resposta das principais funcionalidades do sistema e a taxa de requisições processadas por segundo. Os resultados do tempo de resposta de uma requisição ou query serão obtidos através das ferramentas de monitoramento DataDog e Sentry. Também será registrada a utilização média de CPU e memória durante os testes, utilizando os gráficos fornecidos pelo provedor de serviços em nuvem (AWS).
		\\\hline

		\parbox[t]{2cm}{\raggedleft Critério de
		\\Aceitação:} &  O tempo médio de resposta das operações principais (consulta, cadastro e atualização) não deve exceder 2 segundos sob carga normal de uso e 5 segundos em situações de pico. O sistema deve suportar pelo menos 100 requisições simultâneas sem apresentar degradação significativa de desempenho ou indisponibilidade.\\\hline

		% Linha em branco.
		\multicolumn{2}{c}{}
		\\\hline

		% Requisito não funcional 5.
		\multicolumn{2}{|p{\dimexpr\linewidth-2\tabcolsep-2\arrayrulewidth}|}{\cellcolor{lightgray}\RNF\label{RNF-05} -- O sistema deve ter compatibilidade com diferentes sistemas operacionais}
		\\\hline
		%Compatibilidade/Portabilidade
		Categoria: & Compatibilidade
		\\\hline

		\parbox[t]{2cm}{\raggedleft Tática /
		\\Tratamento:} & A aplicação deve ser desenvolvida utilizando tecnologias web padrão (HTML5, CSS3, JavaScript) e frameworks modernos que garantam compatibilidade com os principais navegadores (Google Chrome, Mozilla Firefox, Microsoft Edge, Safari) em diferentes versões. Deve-se utilizar bibliotecas e ferramentas que facilitem a adaptação da interface para diferentes resoluções de tela e dispositivos.\\\hline

		Medida:    & A compatibilidade deve ser verificada por meio de testes de execução da aplicação em cada navegador-alvo, registrando-se eventuais falhas específicas de cada ambiente e versão através da ferramenta de detecção de erros Sentry.                                                                                                                                                                                                                                                                      \\\hline

		\parbox[t]{2cm}{\raggedleft Critério de
		\\Aceitação:} & As ações do usuário nunca devem resultar em erros na interface da aplicação (front-end) devido imcompatibilidades dos frameworks e bibliotecas para diferentes navegadores e suas versões. Qualquer erro detectado deve ter origem apenas nas camadas inferiores, como na lógica de negócio (que se deve ficar no backend) ou no banco de dados. \\\hline

		% Linha em branco.
		\multicolumn{2}{c}{}
		\\\hline

		% Requisito não funcional 6.
		\multicolumn{2}{|p{\dimexpr\linewidth-2\tabcolsep-2\arrayrulewidth}|}{\cellcolor{lightgray}\RNF\label{RNF-06} --O código-fonte deve ser bem documentado para facilitar a manutenção futura e devem ser implementadas boas práticas de desenvolvimento de software para facilitar a extensibilidade }
		\\\hline

		Categoria: & Manutenibilidade
		\\\hline

		\parbox[t]{2cm}{\raggedleft Tática /
		\\Tratamento:} &  O código deve ser estruturado de forma modular, com alta coesão e baixo acoplamento entre componentes, facilitando alterações e extensões futuras. Cada módulo deve conter documentação clara sobre sua função, entradas e saídas, bem como instruções de uso e integração. Devem ser adotadas convenções de codificação, padronização de nomenclaturas e padrões de projeto que favoreçam a reutilização e a testabilidade do sistema.\\\hline

		Medida:    & A manutenibilidade será avaliada por meio da revisão de código (code review), utilizando a ferramenta Sonarqube e da análise de métricas de qualidade, como o grau de acoplamento entre módulos, a complexidade e a cobertura de testes. Também será observada a porcentagem de classes e métodos documentados em relação ao total do código-fonte, bem como o tempo médio necessário para compreender e modificar uma funcionalidade existente durante atividades de manutenção.                       \\\hline

		\parbox[t]{2cm}{\raggedleft Critério de
		\\Aceitação:} & Cada módulo deve ter funções bem definidas e baixo acoplamento — alterações em um módulo não devem exigir mudanças em mais de um outro módulo.
		O código será avaliado pelo Sonarqube, uma ferramenta que detecta a complexidade de funções e métodos, como também a clareza da nomeclatura
		Uma modificação simples (como ajustar uma regra de negócio) deve poder ser compreendida e alterada em até 2 horas por outro desenvolvedor que não escreveu o código.                                                                                                                                                                                                                                                                                                                                                 \\\hline

		% Linha em branco.
		\multicolumn{2}{c}{}
		\\\hline
	\end{longtable}
\end{footnotesize}

\vspace*{1.3cm}

\chapter{Arquitetura de Software}
\label{sec-arquitetura}
\vspace{-1cm}

A Figura~\ref{figura-arquitetura} mostra a arquitetura do sistema \emph{\imprimirtitulo}.

\begin{figure}[h]
	\centering
	\includegraphics[width=0.5\textwidth,height=0.6\textheight]{figuras/figura-arquitetura.jpg}
	\caption{Arquitetura do sistema.}
	\label{figura-arquitetura}
\end{figure}

\vspace{0.5cm}

A arquitetura do sistema \emph{\imprimirtitulo} é baseada na Arquitetura em Camadas~\cite{tu2023layered}, adaptando a Arquitetura MVC~\cite{qureshi2014mvc}. Ela é organizada em três camadas.
\alvaro{No projeto inicial de Engenharia de Software 1 (até a v 2.0), havíamos adotado o estilo arquitetônico de partições, mas ao longo do desenvolvimento do projeto, percebemos que a arquitetura em camadas se encaixava melhor nas necessidades do sistema. Portanto, o projeto foi alterado para refletir isso.}

As camadas da arquitetura de cada subsistema são definidas como:
\begin{itemize}
	\item \textbf{Camada de Interface com o Usuário (CIU):} Responsável pela comunicação entre o usuário e o sistema, incluindo telas (views) e controladores (controllers) de interação.
	\item \textbf{Camada de Lógica de Negócio (CLN):} Implementa a lógica de aplicação e de domínio do problema, separando a lógica de tarefas (application) da lógica de domínio (CDP, domain) da aplicação.
	\item \textbf{Camada de Domínio do Problema (CDP):} Contém as classes que representam os conceitos e regras do domínio do problema, nesse projeto, está contida dentro da CLN.
	\item \textbf{Camada de Gerência de Dados (CGD):} Responsável pela persistência e recuperação de dados, seguindo o padrão Data Access Object (DAO).
\end{itemize}
A comunicação entre as camadas segue o padrão MVC, onde as páginas do front-end (CIU) chamam os adaptadores responsáveis pelas respostas via protocolo RESTful no back-end, acionando as classes controladoras (CIU), que interagem com as classes da camada de lógica de negócios (CLN) que, por sua vez, manipula as classes do domínio do problema (CDP) e se comunica com a camada de gerência de dados (CGD), a fim de persistir os registros da aplicação.

\vspace{0.5cm}

\vspace*{1.3cm}

\chapter{Projeto dos Componentes da Arquitetura}
\label{sec-componentes}
\vspace{-1cm}

Conforme mostrado na Figura~\ref{figura-arquitetura}, o sistema é separado em camadas e seus pacotes, cada pacote contém seus proprios componentes, que apenas se relacionam com componentes do mesmo pacote ou com componentes de outros pacotes através de interfaces bem definidas.
A seguir, são apresentados os componentes de cada camada do sistema.

\section{Camada de Lógica de Negócio (CLN)}
\label{sec-componentes-cln}

Os componentes da Camada de Lógica de Negócio (CLN) são responsáveis por implementar as regras de negócio do sistema, realizando o processamento dos dados recebidos da camada de apresentação (CIU) e interagindo com a Camada de gerência de dados (CGD) para persistência e recuperação das informações necessárias.

A CLN também contém dois pacotes principais, o pacote Domain e o pacote Application, e interage com a Camada de Domínio do Problema (CDP) através das classes do pacote Domain, para utilizar as entidades específicas do domínio e seus métodos e atributos.

\subsection{Pacote Application}
\label{sec-componentes-cln-application}

O pacote Application é responsável por orquestrar as operações do sistema, coordenando a interação entre os componentes da CLN e a CGD. Ele contém os serviços que implementam as funcionalidades principais do sistema, utilizando as entidades do pacote Domain para realizar as operações de negócio.


\subsection{Pacote Domain}
\label{sec-componentes-cln-domain}

\begin{figure}[h]
	\centering
	\includegraphics[width=1\textwidth]{figuras/diagrama-dominio.jpg}
	\caption{Componentes de Domínio do Problema}
	\label{figura-cln-domain}
\end{figure}

O pacote Domain contém as entidades do domínio do problema, representando os conceitos e objetos relevantes para o sistema. Cada entidade encapsula seus atributos e comportamentos (métodos), fornecendo uma interface clara para a manipulação dos dados relacionados ao domínio.

\vitor{Registrar as principais decisões tomadas para se chegar a este modelo relativas a alterações estruturais do modelo, e.g., aquelas que levaram à criação de novas classes e associações.}

\endgroup



%%% Páginas finais do documento: bibliografia e anexos. %%%
% Finaliza a parte no bookmark do PDF para que se inicie o bookmark na raiz e adiciona espaço de parte no sumário.
\phantompart

% Marca o início dos elementos pós-textuais.
\postextual

% Referências bibliográficas
\bibliography{bibliografia}

% Índice remissivo.
\phantompart
\printindex

% Fim do documento.
\end{document}
